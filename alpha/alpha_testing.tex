\documentclass{article}

%Packages
\usepackage{graphicx}
\usepackage{grffile}
\usepackage{float}

%Margins
\usepackage[
	margin=2cm,
	includefoot
	]{geometry}

%Images
\usepackage{graphicx}

\graphicspath{{images/}}

%Headers and Footers
\usepackage{fancyhdr}
\pagestyle{fancy}
\fancyhead{}
\fancyfoot{}
\fancyfoot[R]{\thepage}
\renewcommand{\headrulewidth}{0pt}
\renewcommand{\footrulewidth}{0pt}


%Details
\title{
Researcher Support System (RSS)
Funcitonal and Achitectural Evaluation and Testing
University of Pretoria
}
\date{2016-04-15}
\author{Team People}

%Document start
\begin{document}

%Title Page
\begin{titlepage}
	\begin{center}
		\includegraphics[width=10cm]{UP.jpg}  \\
		[1cm]
		\line(1,0){300} \\
		[0.3cm]
		\textsc{\Large
			Researcher Support System (RSS)\\
			Functional and Architectural Evaluation and Testing
			%University of Pretoria
		}\\
		[0.1cm]
		\line(1,0){300} \\
		[0.4cm]
		\textsc{\Large
			Team People - Alpha Testing
		} \\



	\end{center}
	\begin{center}

	\textsc{\large\\
	Avinash Singh\\ 
	u14043778\\ 
	}
	
	\textsc{\large\\
	Amy Lochner\\
	u14038600\\ 
	}
	
	\textsc{\large\\
	Themba Mbhele\\
	u14007950\\ 
	}

	\textsc{\large\\
	Unarine Rambani\\
	u14004489 \\
	}
	
	\textsc{\large\\
	Dian Veldsman\\
	u12081095\\
	}
	
	\end{center}
\end{titlepage}

%Table of contents
 
\tableofcontents
\thispagestyle{empty}
\cleardoublepage


%Content
\setcounter{page}{1}
\section{Introduction}
This document notes the conclusions, made by Team people for the testing phase, for Team Alpha's people module with regard to functional and architecture analysis. It documents conclusions made by Team people with regard to the following points:

\begin{itemize}
    \item to what extent the functional requirements as per the functional specification have been met
        \begin{itemize}
            \item the functionality which has been correctly implemented
            \item short-comings of the implemented functionality (with     reference to the pre- and post-conditions)
            \item missing functionality
        \end{itemize}
        
    \item to what extent the module implementation complies to the specified architecture
        \begin{itemize}
            \item which aspects of the software architecture specification were adhered to
            \item which aspects of the software architecture specification were partially complied to 
            \item any short comings with respect to the specification
            \item which aspects of the software architecture were not implemented
        \end{itemize}
\end{itemize}

\section{Functional Testing Report}
	\subsection{Introduction}
	After studying the code, the first conclusion to be arrived at was the fact that functionality was not consistent with the scope provided in the Application Requirements Design Document. We feel that this is bad practice as the design documentation provides a set of standards which should be adhered to, in order to optimise understanding of collaborators on the project.

	\subsection{Correctly implemented Functionality}
	\subsubsection{Scope}
	From the scope provided to Team Alpha People the following functionality was provided:
	\begin{itemize}
		\item addPerson - was found in User.java that followed the correct structure and exception handling was provided for all cases that would be thrown if invalid data was passed through, there is also error checking for invalid email addresses using regex.
		\item endResearchGroupAssociation - was found in Person.java where the function took all the necessary parameters needed to provide the functionality through the information passed through, again there was exception handling which is a good mechanism to detect faults early on in the execution.
		\item addResearchGroupAssociation - found in Person.java also provided exceptions for all cases.
		\item addResearchCategory - found in Person.java also provided exceptions for all cases.
		
	\end{itemize}
	
	\subsubsection{Domain Model}
	From the Domain model provided to Team Alpha People the following structure was provided:
	\begin{itemize}
	 \item ResearchGroupAssociation - which consists of a start date,an end date and the getters and setters.
	 \item ResearchGroupAssociationType - which is an enumeration class containing the type of user, followed the relationship in the model.
	 \item Group - this class contains basic meta data about the group.
	 \item Entity - this was just an empty interface which other classes either extended or implemented, there should have been at least one function in the entity which was not provided.
	 \item User - the class that contained all the information about a user, it contained 'name', 'surname','email' which was needed.
	 \item Organization - was just a simple class that contained meta data about the organization.
	 \item ResearcherCategoryAssociation - was just a simple class that contained the effective date and the type
	 \item ResearcherCategory - has the relationship with ResearcherCategoryStatus and ResearcherCategoryAssociation as a middle-man for communication between them
	 
	\end{itemize}

    \subsection{Implemented Functionality Short-Comings (With Reference To Pre- And Post-Conditions)}
	\begin{itemize}
	
	    \item During the analysis of the people module we found that there was no error checking for null objects which is a potential threat to the system since there could be a failure in object creation, causing the entire system to crash.
	    \item A potential issue is that making a User an Admin is not secure since it is a public function which can be exploited.
	    \item For all the GroupAssociation operations, the implementation was faulty since they added a copy of the class (this) to a list. This is a bad approach and can lead to duplicates as well as data invalidation, thus not satisfying the Reliability aspect of the requirements.
	    \item For the User object, getters and setters are implemented in the PersonImplementation.java. We feel that it should have been implemented separate from Persons since it is supposed to be abstract. 
	    \item The exception classes that were implemented did not cover all of the possible errors that can occur when creating a Person. For example, there were no checks done to verify whether a valid name was provided (it is possible to send in an empty string) and thus there was no exception class that deals with this matter.
	\end{itemize}

    \subsection{Missing Functionality}
	\subsubsection{Scope}
	From the scope provided to Team Alpha People the following functionality was not provided:
	\begin{itemize}
		\item editPerson - No implementation
		\item reactivateResearchGroup - No implementation
		\item suspendResearchGroup - No implementation
		\item addResearchGroup - No implementation
		\item modifyResearchCategory - No implementation
	\end{itemize}
	
	\subsubsection{Domain Model}
	From the Domain model provided to Team Alpha People the following structure was not provided:
	\begin{itemize}
		\item Person - According to the Domain Model there should be a User object in a Person Object which was not implemented correctly since they did not have the object in Person but rather added all the functionality of the User class to Person class which defeats the purpose of creating the User mock class.
		\item EmailAddress - there was no email address class implemented or created anywhere however they just had a string containing the email address.
		\item ResearcherCatogoryStatus - an empty interface with no implementation
	 \end{itemize}

\section{Architecture Compliance Analysis}
	\subsection{Software Architecture Specifications Adherence}
	\begin{itemize}

	\item Persistence was implemented correctly in which they had ID generation, which provides unique access to a specific object so there can be modification, however this module did not implement this and could have been done without much effort. 

			\item In their code they provided well for persistence.
		\item JPA - this was implemented correctly and was consistent, they even created a ORM.xml for the Object Relational Mapping for the database done by the entity manager. They used an in-memory database for testing, which is an easier method and is ideal since it is only used for testing purposes.
	\end{itemize}

	\subsection{Partially Complied Software Architecture Specifications (Pointing Out Any Short-Comings With Respect To The Specification)}

	\begin{itemize}

		\item Unit tests were not implemented correctly and took great difficulty to get working. They were not structured properly and they were not configured to test all aspects of their code. Their unit test was also very trivial, since they used persistence they could have done more testing but they only created 3 Mock objects and put it into the database and only tested if they could access the last one, it could be that the database was incorrectly set up which influenced the overall test.
		\item Although the team had a pom file and a Maven-looking project, they implemented Ant builds for the project instead, which goes against one of the architectural requirements.
	\end{itemize}

	\subsection{Missing Aspects Of The Software Architecture Specifications}
	\begin{itemize}
		\item Dependency injection framework was not implemented
		\item Object-relational mapping was not utilized
		\item There is no sign of resource pooling or resource recycling in the implementation
		\item There is no support for transactions
		\item There is no sign of JPQL qeuries which indicates that query mapping was not done
	\end{itemize}
	

\section{Conclusion}
After studying the code, the first conclusion to be arrived at was the fact that functionality was not consistent with the scope provided in the Application Requirements Design Document. We feel this is a bad practice as the design documentation provides a set of standards which should be adhered to, in order to optimize understanding of collaborators on the project.  Our overall conclusion was that the code was very convoluted and complex in many cases where it could be very simple, easy to understand and easy to implement. The code was not well documented which caused great difficulty for us to understand and figure out. The main interface User.java we felt was very incomplete as it should provide a means to make use of all the functionality specified in the scope which it did not. We felt this group lacked the understanding of how the system operates and how the People module would be incorporated into the system. They did not provide any functionality towards management of Research Groups. They provided management for Researcher Categories in the People interface which supports our first statement of their code being convoluted. They catered well for Membership Management of individual people.

\end{document}